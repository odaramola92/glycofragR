\nonstopmode{}
\documentclass[a4paper]{book}
\usepackage[times,inconsolata,hyper]{Rd}
\usepackage{makeidx}
\makeatletter\@ifl@t@r\fmtversion{2018/04/01}{}{\usepackage[utf8]{inputenc}}\makeatother
% \usepackage{graphicx} % @USE GRAPHICX@
\makeindex{}
\begin{document}
\chapter*{}
\begin{center}
{\textbf{\huge Package `glycofragR'}}
\par\bigskip{\large \today}
\end{center}
\ifthenelse{\boolean{Rd@use@hyper}}{\hypersetup{pdftitle = {glycofragR: Glycan and Glycopeptide Fragmentation Analysis via Glycofrag}}}{}
\ifthenelse{\boolean{Rd@use@hyper}}{\hypersetup{pdfauthor = {Oluwatosin Daramola}}}{}
\begin{description}
\raggedright{}
\item[Type]\AsIs{Package}
\item[Title]\AsIs{Glycan and Glycopeptide Fragmentation Analysis via Glycofrag}
\item[Version]\AsIs{0.1.0}
\item[Author]\AsIs{Oluwatosin Daramola [aut, cre]}
\item[Maintainer]\AsIs{Oluwatosin Daramola }\email{oluwatosindaramola@gmail.com}\AsIs{}
\item[Description]\AsIs{R wrapper for the glycofrag Python package. Provides glycan and
glycopeptide structure prediction, fragmentation, visualization, and mass
calculations through reticulate.}
\item[License]\AsIs{MIT + file LICENSE}
\item[Encoding]\AsIs{UTF-8}
\item[LazyData]\AsIs{true}
\item[Imports]\AsIs{reticulate}
\item[Suggests]\AsIs{testthat (>= 3.0.0), knitr, rmarkdown}
\item[VignetteBuilder]\AsIs{knitr}
\item[Config/testthat/edition]\AsIs{3}
\item[Roxygen]\AsIs{list(markdown = TRUE)}
\item[RoxygenNote]\AsIs{7.3.2}
\end{description}
\Rdcontents{Contents}
\HeaderA{glycan\_analysis\_clean\_df}{Get clean theoretical fragments from GlycanAnalysis}{glycan.Rul.analysis.Rul.clean.Rul.df}
%
\begin{Description}
Get clean theoretical fragments from GlycanAnalysis
\end{Description}
%
\begin{Usage}
\begin{verbatim}
glycan_analysis_clean_df(analysis)
\end{verbatim}
\end{Usage}
%
\begin{Arguments}
\begin{ldescription}
\item[\code{analysis}] GlycanAnalysis object
\end{ldescription}
\end{Arguments}
%
\begin{Value}
Data frame of unique/deduplicated fragments
\end{Value}
\HeaderA{glycan\_analysis\_export}{Export GlycanAnalysis to Excel}{glycan.Rul.analysis.Rul.export}
%
\begin{Description}
Export GlycanAnalysis to Excel
\end{Description}
%
\begin{Usage}
\begin{verbatim}
glycan_analysis_export(analysis, output_file, include_prediction = TRUE)
\end{verbatim}
\end{Usage}
%
\begin{Arguments}
\begin{ldescription}
\item[\code{analysis}] GlycanAnalysis object

\item[\code{output\_file}] Path to output Excel file

\item[\code{include\_prediction}] Include structure prediction sheet (default: TRUE)
\end{ldescription}
\end{Arguments}
%
\begin{Value}
NULL invisibly
\end{Value}
%
\begin{Examples}
\begin{ExampleCode}
## Not run: 
set_glycofrag_conda_env("glycofrag")
analysis <- glycan_analysis_new("4501", "N")
glycan_analysis_export(analysis, "output.xlsx")

## End(Not run)
\end{ExampleCode}
\end{Examples}
\HeaderA{glycan\_analysis\_new}{GlycanAnalysis wrapper}{glycan.Rul.analysis.Rul.new}
%
\begin{Description}
High-level interface for glycan analysis combining structure prediction,
fragment generation, and Excel export.
\end{Description}
%
\begin{Usage}
\begin{verbatim}
glycan_analysis_new(
  glycan_code,
  glycan_type = "N",
  modification_type = "permethylated_reduced",
  max_structures = 100,
  isomer_sensitive = FALSE,
  fragment_types = c("BY"),
  charges = c(1, 2, 3),
  visualize_structure = NULL,
  preferred_core = NULL
)
\end{verbatim}
\end{Usage}
%
\begin{Arguments}
\begin{ldescription}
\item[\code{glycan\_code}] 4-5 digit composition code

\item[\code{glycan\_type}] Type of glycan: "N" or "O" (default: "N")

\item[\code{modification\_type}] Reducing end modification (default: "permethylated\_reduced")

\item[\code{max\_structures}] Maximum number of structures to predict (default: 100)

\item[\code{isomer\_sensitive}] Treat mirror images as distinct (default: FALSE)

\item[\code{fragment\_types}] Vector of fragment types (default: c("BY"))

\item[\code{charges}] Vector of charge states (default: c(1, 2, 3))

\item[\code{visualize\_structure}] Which structures to visualize: "all", "best", integer, or NULL

\item[\code{preferred\_core}] For O-glycans, select specific core (1-8) or NULL
\end{ldescription}
\end{Arguments}
%
\begin{Value}
GlycanAnalysis object
\end{Value}
%
\begin{Examples}
\begin{ExampleCode}
## Not run: 
set_glycofrag_conda_env("glycofrag")
analysis <- glycan_analysis_new(
  glycan_code = "4501",
  glycan_type = "N",
  modification_type = "permethylated_reduced"
)

## End(Not run)
\end{ExampleCode}
\end{Examples}
\HeaderA{glycan\_analysis\_structures}{Get GlycanAnalysis structures}{glycan.Rul.analysis.Rul.structures}
%
\begin{Description}
Get GlycanAnalysis structures
\end{Description}
%
\begin{Usage}
\begin{verbatim}
glycan_analysis_structures(analysis)
\end{verbatim}
\end{Usage}
%
\begin{Arguments}
\begin{ldescription}
\item[\code{analysis}] GlycanAnalysis object
\end{ldescription}
\end{Arguments}
%
\begin{Value}
List of predicted structures
\end{Value}
\HeaderA{glycan\_analysis\_summary}{Get summary from GlycanAnalysis}{glycan.Rul.analysis.Rul.summary}
%
\begin{Description}
Get summary from GlycanAnalysis
\end{Description}
%
\begin{Usage}
\begin{verbatim}
glycan_analysis_summary(analysis)
\end{verbatim}
\end{Usage}
%
\begin{Arguments}
\begin{ldescription}
\item[\code{analysis}] GlycanAnalysis object
\end{ldescription}
\end{Arguments}
%
\begin{Value}
Data frame with analysis summary
\end{Value}
\HeaderA{glycan\_analysis\_theoretical\_df}{Get theoretical fragments from GlycanAnalysis}{glycan.Rul.analysis.Rul.theoretical.Rul.df}
%
\begin{Description}
Get theoretical fragments from GlycanAnalysis
\end{Description}
%
\begin{Usage}
\begin{verbatim}
glycan_analysis_theoretical_df(analysis)
\end{verbatim}
\end{Usage}
%
\begin{Arguments}
\begin{ldescription}
\item[\code{analysis}] GlycanAnalysis object
\end{ldescription}
\end{Arguments}
%
\begin{Value}
Data frame of theoretical fragments
\end{Value}
\HeaderA{glycan\_analysis\_visualize}{Visualize structure from GlycanAnalysis}{glycan.Rul.analysis.Rul.visualize}
%
\begin{Description}
Visualize structure from GlycanAnalysis
\end{Description}
%
\begin{Usage}
\begin{verbatim}
glycan_analysis_visualize(
  analysis,
  structure_number,
  output_path = NULL,
  show = FALSE,
  show_node_numbers = TRUE,
  ...
)
\end{verbatim}
\end{Usage}
%
\begin{Arguments}
\begin{ldescription}
\item[\code{analysis}] GlycanAnalysis object

\item[\code{structure\_number}] Which structure to visualize (1-based)

\item[\code{output\_path}] Path to save image

\item[\code{show}] Display the plot (default: FALSE)

\item[\code{show\_node\_numbers}] Show node numbers (default: TRUE)

\item[\code{...}] Additional visualization parameters
\end{ldescription}
\end{Arguments}
%
\begin{Value}
NULL invisibly
\end{Value}
\HeaderA{glycan\_generate\_fragments}{Generate glycan fragments}{glycan.Rul.generate.Rul.fragments}
%
\begin{Description}
Generate theoretical fragment ions for a glycan structure.
\end{Description}
%
\begin{Usage}
\begin{verbatim}
glycan_generate_fragments(
  glycan,
  structure,
  modification_type = 0,
  permethylated = FALSE,
  fragment_types = c("BY", "CZ")
)
\end{verbatim}
\end{Usage}
%
\begin{Arguments}
\begin{ldescription}
\item[\code{glycan}] Glycan object

\item[\code{structure}] Structure graph from glycan\_predict\_structures()

\item[\code{modification\_type}] Reducing end modification type (0-6, default: 0)

\item[\code{permethylated}] Logical; if TRUE, use permethylated masses (default: FALSE)

\item[\code{fragment\_types}] Vector of fragment types (default: c("BY", "CZ"))
\end{ldescription}
\end{Arguments}
%
\begin{Value}
List containing fragments and metadata
\end{Value}
%
\begin{Examples}
\begin{ExampleCode}
## Not run: 
set_glycofrag_conda_env("glycofrag")
glycan <- glycan_new("4501", "N")
structures <- glycan_predict_structures(glycan)
frags <- glycan_generate_fragments(glycan, structures[[1]])

## End(Not run)
\end{ExampleCode}
\end{Examples}
\HeaderA{glycan\_mass}{Calculate glycan mass}{glycan.Rul.mass}
%
\begin{Description}
Calculate mass from a glycan composition code.
\end{Description}
%
\begin{Usage}
\begin{verbatim}
glycan_mass(glycan_code)
\end{verbatim}
\end{Usage}
%
\begin{Arguments}
\begin{ldescription}
\item[\code{glycan\_code}] 4-5 digit composition code
\end{ldescription}
\end{Arguments}
%
\begin{Value}
Numeric mass in Daltons
\end{Value}
%
\begin{Examples}
\begin{ExampleCode}
## Not run: 
set_glycofrag_conda_env("glycofrag")
mass <- glycan_mass("4501")

## End(Not run)
\end{ExampleCode}
\end{Examples}
\HeaderA{glycan\_new}{Create a new Glycan object}{glycan.Rul.new}
%
\begin{Description}
Create a new Glycan object
\end{Description}
%
\begin{Usage}
\begin{verbatim}
glycan_new(glycan_code, glycan_type = "N")
\end{verbatim}
\end{Usage}
%
\begin{Arguments}
\begin{ldescription}
\item[\code{glycan\_code}] 4-5 digit composition code (e.g., "4501" for HexNAc-4, Hex-5, Fuc-0, NeuAc-1)

\item[\code{glycan\_type}] Type of glycan: "N" for N-glycan or "O" for O-glycan (default: "N")
\end{ldescription}
\end{Arguments}
%
\begin{Value}
Python Glycan object
\end{Value}
%
\begin{Examples}
\begin{ExampleCode}
## Not run: 
set_glycofrag_conda_env("glycofrag")
glycan <- glycan_new("4501", "N")

## End(Not run)
\end{ExampleCode}
\end{Examples}
\HeaderA{glycan\_predict\_structures}{Predict glycan structures}{glycan.Rul.predict.Rul.structures}
%
\begin{Description}
Predict possible structures from a glycan composition.
\end{Description}
%
\begin{Usage}
\begin{verbatim}
glycan_predict_structures(glycan)
\end{verbatim}
\end{Usage}
%
\begin{Arguments}
\begin{ldescription}
\item[\code{glycan}] Glycan object created by glycan\_new()
\end{ldescription}
\end{Arguments}
%
\begin{Value}
List of predicted structure graphs
\end{Value}
%
\begin{Examples}
\begin{ExampleCode}
## Not run: 
set_glycofrag_conda_env("glycofrag")
glycan <- glycan_new("4501", "N")
structures <- glycan_predict_structures(glycan)

## End(Not run)
\end{ExampleCode}
\end{Examples}
\HeaderA{glycan\_visualize}{Visualize glycan structure}{glycan.Rul.visualize}
%
\begin{Description}
Create a SNFG-compliant visualization of a glycan structure.
\end{Description}
%
\begin{Usage}
\begin{verbatim}
glycan_visualize(
  structure,
  title = NULL,
  output_path = NULL,
  show = FALSE,
  ...
)
\end{verbatim}
\end{Usage}
%
\begin{Arguments}
\begin{ldescription}
\item[\code{structure}] Structure graph from glycan\_predict\_structures()

\item[\code{title}] Optional title for the plot

\item[\code{output\_path}] Optional file path to save the image

\item[\code{show}] Logical; if TRUE, display the plot (default: FALSE)

\item[\code{...}] Additional visualization parameters (vertical\_gap, horizontal\_spacing, node\_size, figsize)
\end{ldescription}
\end{Arguments}
%
\begin{Value}
NULL invisibly
\end{Value}
%
\begin{Examples}
\begin{ExampleCode}
## Not run: 
set_glycofrag_conda_env("glycofrag")
glycan <- glycan_new("4501", "N")
structures <- glycan_predict_structures(glycan)
glycan_visualize(structures[[1]], output_path = "glycan.png")

## End(Not run)
\end{ExampleCode}
\end{Examples}
\HeaderA{glycofragR}{glycofragR}{glycofragR}
\aliasA{glycofragR-package}{glycofragR}{glycofragR.Rdash.package}
%
\begin{Description}
R wrapper for the glycofrag Python package.
\end{Description}
%
\begin{Author}
\strong{Maintainer}: Oluwatosin Daramola \email{oluwatosindaramola@gmail.com}

\end{Author}
\HeaderA{glycopeptide\_fragments}{Generate glycopeptide fragments}{glycopeptide.Rul.fragments}
%
\begin{Description}
Generate all theoretical fragment ions for a glycopeptide.
\end{Description}
%
\begin{Usage}
\begin{verbatim}
glycopeptide_fragments(glycopeptide)
\end{verbatim}
\end{Usage}
%
\begin{Arguments}
\begin{ldescription}
\item[\code{glycopeptide}] Glycopeptide object created by glycopeptide\_new()
\end{ldescription}
\end{Arguments}
%
\begin{Value}
List of fragment ions (peptide, glycan, and glycopeptide fragments)
\end{Value}
%
\begin{Examples}
\begin{ExampleCode}
## Not run: 
set_glycofrag_conda_env("glycofrag")
gp <- glycopeptide_new("LCPDCPLLAPLNDSR", "4501", 12, "N")
frags <- glycopeptide_fragments(gp)

## End(Not run)
\end{ExampleCode}
\end{Examples}
\HeaderA{glycopeptide\_new}{Create a new Glycopeptide object}{glycopeptide.Rul.new}
%
\begin{Description}
Create a new Glycopeptide object
\end{Description}
%
\begin{Usage}
\begin{verbatim}
glycopeptide_new(
  peptide_sequence,
  glycan_code,
  glycosylation_site,
  glycan_type = "N",
  modifications = NULL,
  mod_string = NULL
)
\end{verbatim}
\end{Usage}
%
\begin{Arguments}
\begin{ldescription}
\item[\code{peptide\_sequence}] Amino acid sequence

\item[\code{glycan\_code}] 4-5 digit glycan composition code

\item[\code{glycosylation\_site}] Position of glycosylation (1-based indexing)

\item[\code{glycan\_type}] Type of glycan: "N" or "O" (default: "N")

\item[\code{modifications}] Optional named list of peptide modifications with positions (1-based)

\item[\code{mod\_string}] Optional modification string (e.g., "M:Ox; C:CAM")
\end{ldescription}
\end{Arguments}
%
\begin{Value}
Python Glycopeptide object
\end{Value}
%
\begin{Examples}
\begin{ExampleCode}
## Not run: 
set_glycofrag_conda_env("glycofrag")
gp <- glycopeptide_new(
  peptide_sequence = "LCPDCPLLAPLNDSR",
  glycan_code = "4501",
  glycosylation_site = 12,
  glycan_type = "N"
)

## End(Not run)
\end{ExampleCode}
\end{Examples}
\HeaderA{glycopeptide\_summary}{Get glycopeptide summary}{glycopeptide.Rul.summary}
%
\begin{Description}
Extract key information from a glycopeptide object.
\end{Description}
%
\begin{Usage}
\begin{verbatim}
glycopeptide_summary(glycopeptide)
\end{verbatim}
\end{Usage}
%
\begin{Arguments}
\begin{ldescription}
\item[\code{glycopeptide}] Glycopeptide object
\end{ldescription}
\end{Arguments}
%
\begin{Value}
List with peptide\_sequence, glycan\_code, glycosylation\_site, glycan\_type
\end{Value}
%
\begin{Examples}
\begin{ExampleCode}
## Not run: 
set_glycofrag_conda_env("glycofrag")
gp <- glycopeptide_new("LCPDCPLLAPLNDSR", "4501", 12, "N")
summary <- glycopeptide_summary(gp)

## End(Not run)
\end{ExampleCode}
\end{Examples}
\HeaderA{list\_supported\_modifications}{List supported modifications}{list.Rul.supported.Rul.modifications}
%
\begin{Description}
Get a list of all supported peptide modifications.
\end{Description}
%
\begin{Usage}
\begin{verbatim}
list_supported_modifications(verbose = TRUE)
\end{verbatim}
\end{Usage}
%
\begin{Arguments}
\begin{ldescription}
\item[\code{verbose}] Logical; if TRUE, print detailed information (default: TRUE)
\end{ldescription}
\end{Arguments}
%
\begin{Value}
List of supported modifications with masses and targets
\end{Value}
%
\begin{Examples}
\begin{ExampleCode}
## Not run: 
set_glycofrag_conda_env("glycofrag")
mods <- list_supported_modifications()

## End(Not run)
\end{ExampleCode}
\end{Examples}
\HeaderA{peptide\_fragments}{Generate peptide fragments}{peptide.Rul.fragments}
%
\begin{Description}
Generate theoretical peptide fragment ions (b, y, c, z).
\end{Description}
%
\begin{Usage}
\begin{verbatim}
peptide_fragments(peptide, charge_states = c(1, 2))
\end{verbatim}
\end{Usage}
%
\begin{Arguments}
\begin{ldescription}
\item[\code{peptide}] Peptide object created by peptide\_new()

\item[\code{charge\_states}] Vector of charge states to generate (default: c(1, 2))
\end{ldescription}
\end{Arguments}
%
\begin{Value}
List of fragment ions
\end{Value}
%
\begin{Examples}
\begin{ExampleCode}
## Not run: 
set_glycofrag_conda_env("glycofrag")
peptide <- peptide_new("PEPTIDE")
frags <- peptide_fragments(peptide, charge_states = c(1, 2, 3))

## End(Not run)
\end{ExampleCode}
\end{Examples}
\HeaderA{peptide\_new}{Create a new Peptide object}{peptide.Rul.new}
%
\begin{Description}
Create a new Peptide object
\end{Description}
%
\begin{Usage}
\begin{verbatim}
peptide_new(sequence, modifications = NULL, mod_string = NULL)
\end{verbatim}
\end{Usage}
%
\begin{Arguments}
\begin{ldescription}
\item[\code{sequence}] Amino acid sequence (single-letter code)

\item[\code{modifications}] Optional named list of modifications with positions (1-based)

\item[\code{mod\_string}] Optional modification string (e.g., "M:Ox; S3:Phos")
\end{ldescription}
\end{Arguments}
%
\begin{Value}
Python Peptide object
\end{Value}
%
\begin{Examples}
\begin{ExampleCode}
## Not run: 
set_glycofrag_conda_env("glycofrag")
peptide <- peptide_new("PEPTIDE")
peptide_mod <- peptide_new("LCPDCPLLAPLNDSR", mod_string = "C:CAM")

## End(Not run)
\end{ExampleCode}
\end{Examples}
\HeaderA{set\_glycofrag\_conda\_env}{Set glycofrag conda environment}{set.Rul.glycofrag.Rul.conda.Rul.env}
%
\begin{Description}
Configure reticulate to use a specific conda environment containing glycofrag.
\end{Description}
%
\begin{Usage}
\begin{verbatim}
set_glycofrag_conda_env(env_name = "glycofrag")
\end{verbatim}
\end{Usage}
%
\begin{Arguments}
\begin{ldescription}
\item[\code{env\_name}] Name of the conda environment (default: "glycofrag")
\end{ldescription}
\end{Arguments}
%
\begin{Value}
TRUE invisibly
\end{Value}
%
\begin{Examples}
\begin{ExampleCode}
## Not run: 
set_glycofrag_conda_env("glycofrag")

## End(Not run)
\end{ExampleCode}
\end{Examples}
\printindex{}
\end{document}
